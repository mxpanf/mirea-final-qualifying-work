\section{Аналитический раздел}
\label{sec:analytics}

\subsection{Актуальность и проблематика цифровой идентификации дипломов}

В современном обществе, дипломы об образовательных достижениях и квалификации служат основным подтверждением учебных успехов и профессиональной компетентности личности. Они являются ключевым фактором при трудоустройстве, продвижении по карьерной лестнице и во многих других аспектах профессиональной и социальной жизни. Однако, проблемы, такие как: подделка, утеря или долгий процесс верификации дипломов, становятся всё более актуальными.

Подделка дипломов: Доступность технологий, способствующих фальсификации документов, усиливает риск появления недостоверных дипломов, что осложняет процесс трудоустройства и повышает риски для работодателей и общества. Фальшивые дипломы угрожают не только репутации образовательных учреждений, но и негативно влияют на качество специалистов в различных отраслях, что может иметь долгосрочные негативные последствия для экономики и общества в целом.

Верификация дипломов: Процедуры проверки подлинности дипломов часто оказываются времязатратными и сложными, что создаёт дополнительные трудности для выпускников и работодателей. Такие действия могут охватывать сверку с базами данных, запросы в учебные заведения и другие методы, направленные на исключение возможности использования фальсифицированных данных. Этот процесс может создавать дополнительные преграды для выпускников в пути к трудоустройству, а также увеличивать нагрузку на работодателей в процессе отбора кандидатов.

Утеря документов: В случае утраты или повреждения дипломов возникают проблемы, связанные с восстановлением утраченной информации, что может повлиять на карьерные перспективы выпускников.  В случае утраты диплома, выпускнику приходится сталкиваться с дополнительными сложностями и бюрократическими процедурами по его восстановлению. Это не только затрудняет процесс трудоустройства, но и может негативно сказаться на репутации выпускника в глазах потенциального работодателя.

Введение системы токенизации на основе блокчейн-технологий может стать решением вышеуказанных проблем. Токенизация дипломов позволяет обеспечить уникальный невзаимозаменяемый цифровой слепок для каждого диплома, что существенно затруднит процесс их подделки. Блокчейн, как технология, обеспечивающая децентрализованное хранение и верификацию данных, способна упростить процесс проверки подлинности дипломов, сделав его более прозрачным и безопасным. Это также минимизирует риск утери важных данных, так как информация будет защищена и постоянно доступна в цепочке блоков.

Эти нововведения могут сделать систему высшего образования более надежной и адаптированной к современным вызовам, способствуя более эффективному трудоустройству выпускников и укреплению доверия со стороны работодателей и образовательных учреждений.

\subsection{Применение технологий блокчейна в сфере образования}

Распределенные реестры и блокчейн-технологии открывают новые горизонты в сфере образования, обеспечивая децентрализованное, прозрачное и безопасное хранение данных об образовательных достижениях и квалификациях.

Блокчейн, как разновидность технологии распределенных реестров (Distributed Ledger Technology, DLT), базируется на принципах децентрализации и коллективного подтверждения валидности данных. В блокчейне данные хранятся в цепочке блоков, защищенных криптографическими методами. Такой подход обеспечивает устойчивость к фальсификации и несанкционированному доступу, что может сыграть ключевую роль в управлении образовательными документами. Технология обеспечивает надежное хранение данных, облегчает процесс их верификации и уменьшает риски мошенничества. Таким образом, блокчейн способствует повышению доверия к образовательным учреждениям и их документам.

Использование блокчейна в образовании находится в начальной стадии, но уже сейчас видны перспективы его развития:  распределённый реестр может стать базой для создания унифицированных образовательных платформ, обеспечивающих интероперабельность систем, автоматизацию процессов и глобальную доступность образовательных ресурсов.

Примеры внедрения блокчейн-технологий в образование можно найти по всему миру. Например, в MIT загружают в блокчейн официальные документы, такие как научные работы и открытия для надёжного хранения и неизменности документов.

Другим примером успешного внедрения технологий служит успешное применение блокчейна для проведения тестирования и экзаменов в некоторых университетах для отслеживания попыток прохождения форм оценки знаний и результатов.

Применение цепочки блоков в организации учебного процесса открывает новые возможности: он помогает управлять доступом к библиотечным ресурсам, автоматизировать учетные системы и обеспечивать прозрачность использования ресурсов. Регистрация исследовательских данных и публикаций в блокчейн-системе способствует подтверждению их подлинности, обеспечивает прозрачность и защиту интеллектуальной собственности.

Блокчейн также может быть применен для упрощения и автоматизации финансовых операций, например, оплаты обучения, получения грантов и стипендий. С его помощью можно эффективно вести управление карьерой, предоставляя студентам и специалистам инструменты для документирования и верификации своих навыков и достижений на протяжении всей карьеры, включая, в частности, верификацию дипломов.

Помимо вышеперечисленного, цепочка блоков может служить надежным инструментом для хранения и верификации дипломов, сертификатов и прочих образовательных документов, снижая риск мошенничества с документами. Академические транскрипты и образовательная история студента могут быть зафиксированы в блокчейне, обеспечивая удобный доступ и трансфер данных между учебными заведениями. Блокчейн гарантирует безопасное и прозрачное хранение, а также обмен данными между студентами и преподавателями, что способствует эффективному управлению данными и соблюдению стандартов конфиденциальности.

\subsection{Блокчейн-технологии и умные контракты для управления образовательными достижениями}

\subsubsection{Виды блокчейнов и их применение}

\subsubsection{Преимущества и недостатки блокчейн-технологий}

\subsubsection{Смарт-контракты}

\subsection{Инструменты токенизации и виды токенов}

\subsection{Система децентрализованного хранения данных (IPFS)}

\subsection{Анализ существующих систем}