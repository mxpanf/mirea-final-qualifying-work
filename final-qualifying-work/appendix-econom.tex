\section{Результаты бальной оценки программной разработки <<ПК Дипломер>>}
\label{ap:ocenka}

\begin{longtable}{ x{4cm}x{2cm}x{2.8cm}x{2.8cm}x{2cm} } 
	
    \caption {Результаты бальной оценки ПО по функциональным возможностям}\label{tab:func_cap}\\ \toprule
\endfirsthead
\caption* {Продолжение таблицы \ref{tab:func_cap}}\\ \toprule
	
	\endhead
	
	\endfoot
	
	\endlastfoot
	
	\textbf{Функциональные возможности} & \multicolumn{3}{c}{\textbf{Исследуемые Программные продукты}} & \textbf{Вес критерия} \\ \cmidrule(lr){2-4}
	& \textbf{ПП} & \textbf{ПЕРВЫЙ КОНКУРЕНТ} & \textbf{ВТОРОЙ КОНКУРЕНТ} &                   \\ \midrule
	\multicolumn{5}{@{}l@{}}{\textbf{Пригодность}}                                                                                                               \\
	соответствие назначения целям применения ПС                              & 5                      & 3                       & 4                       & 5                 \\
	соответствие требований к функциям назначению ПС                         & 4                      & 4                       & 3                       & 5                 \\
	соответствие исходной информации требованиям к функциям ПС               & 4                      & 5                       & 2                       & 5                 \\
	соответствие состава и содержания выходной информации для потребителей ПС & 5                    & 3                       & 3                       & 5                 \\
	соответствие структурных характеристик комплекса программ ПС             & 5                    & 4                       & 3                       & 5                 \\
	\midrule\multicolumn{5}{@{}l@{}}{\textbf{Корректность (правильность)}}                                                                                              \\
	соответствие требований к функциям ПС информационной системе              & 5                    & 5                       & 3                       & 2                 \\
	соответствие требований к функциональным компонентам функциям ПС         & 4                    & 5                       & 1                       & 2                 \\
	соответствие текстов программ требованиям к функциональным компонентам   & 4                    & 4                       & 2                       & 2                 \\
	соответствие объектного кода исходному тексту программ                   & 4                    & 5                       & 1                       & 2                 \\
	степень покрытия тестами возможных маршрутов исполнения программ          & 2                    & 4                       & 3                       & 2                 \\
	\midrule\multicolumn{5}{@{}l@{}}{\textbf{Способность к взаимодействию (совместимости)}}                                                                            \\
	с операционной системой                                                  & 2                      & 5                       & 4                       & 5                 \\
	с аппаратной средой                                                      & 4                      & 3                       & 1                       & 5                 \\
	с внешней средой информационной системы и пользователями                 & 3                      & 4                       & 2                       & 5                 \\
	между программными компонентами                                          & 3                      & 4                       & 1                       & 5                 \\
	между компонентами распределенных информационных систем                  & 2                      & 4                       & 5                       & 5                 \\
	\midrule\multicolumn{5}{@{}l@{}}{\textbf{Защищенность}}                                                                                                            \\
	аутентификация элементов систем обработки данных                         & 4                      & 5                       & 1                       & 8                 \\
	управление доступом                                                      & 3                      & 3                       & 3                       & 8                 \\
	протоколирование обращений                                               & 4                      & 3                       & 3                       & 8                 \\
	криптографическая защита                                                 & 5                      & 3                       & 4                       & 8                 \\
	превентивное реагирование                                                & 2                      & 3                       & 3                       & 8                 \\
	\midrule
	\textbf{Сумма баллов}                                                    & \textbf{367}           & \textbf{377}            & \textbf{272}            &                   \\
	\bottomrule
\end{longtable}

\begin{table}[H]
	\caption{Результаты бальной оценки ПО по субъективным пользовательским предпочтениям}
	\centering
	
	\tolerance=0
	\emergencystretch=10pt
	\hyphenpenalty=0
	\exhyphenpenalty=0
	\begin{tabular}{x{4cm}x{2cm}x{2.8cm}x{2.8cm}x{2cm}}
		\toprule

        \textbf{Субъективные пользовательские характеристики} & \multicolumn{3}{c}{\textbf{Исследуемые Программные продукты}} & \textbf{Вес критерия} \\ \cmidrule(lr){2-4}
        & \textbf{ПП} & \textbf{ПЕРВЫЙ КОНКУРЕНТ} & \textbf{ВТОРОЙ КОНКУРЕНТ} &                   \\ \midrule
		возможность настройки ПП                              & 5                                & 3                                        & 1                                        & 50                    \\
		возможность работы с территориально-распределенными офисами & 5                          & 3                                        & 2                                        & 25                    \\
		возможность обработки документов                      & 2                                & 4                                        & 4                                        & 25                    \\
		\textbf{Сумма баллов}                                 & \textbf{175}                     & \textbf{175}                             & \textbf{150}                             &                       \\ \midrule
		\textbf{Индекс конкурентоспособности по субъективным пользовательским предпочтениям} & \textbf{1,00} & \textbf{1,00}       & \textbf{0,86}                            &                       \\
		\bottomrule
	\end{tabular}
	\label{tab:user_char}
\end{table}

\begin{table}[H]
	\caption{Результаты бальной оценки ПО по организационным критериям}
	\centering
	
	\tolerance=0
	\emergencystretch=10pt
	\hyphenpenalty=0
	\exhyphenpenalty=0
	\begin{tabular}{x{4cm}x{2cm}x{2.8cm}x{2.8cm}x{2cm}}
		\toprule

        \textbf{Организационные критерии} & \multicolumn{3}{c}{\textbf{Исследуемые Программные продукты}} & \textbf{Вес критерия} \\ \cmidrule(lr){2-4}
        & \textbf{ПП} & \textbf{ПЕРВЫЙ КОНКУРЕНТ} & \textbf{ВТОРОЙ КОНКУРЕНТ} &                   \\ \midrule
		Система скидок & 4                                & 4                                        & 1                                        & 20                    \\
		Условия платежей и поставок & 4                          & 3                                        & 4                                        & 50                    \\
		Сроки и условия гарантии & 5                                & 1                                        & 3                                        & 30                    \\
		\textbf{Сумма баллов}                                 & \textbf{350}                     & \textbf{180}                             & \textbf{290}                             &                       \\ \midrule
		\textbf{Индекс конкурентоспособности по по организационным критериям} & \textbf{1,00} & \textbf{0,51}       & \textbf{0,83}                            &                       \\
		\bottomrule
	\end{tabular}
	\label{tab:org_char}
\end{table}

\begin{table}[H]
	\caption{Результаты оценки ПО  экономическим критериям (по цене потребления)}
	\centering
	
	\tolerance=0
	\emergencystretch=10pt
	\hyphenpenalty=0
	\exhyphenpenalty=0
	\begin{tabular}{x{4cm}x{2cm}x{2.8cm}x{2.8cm}x{2cm}}
		\toprule

        \textbf{Элементы цены потребления} & \multicolumn{3}{c}{\textbf{Исследуемые Программные продукты}} & \textbf{Вес критерия} \\ \cmidrule(lr){2-4}
        & \textbf{ПП} & \textbf{ПЕРВЫЙ КОНКУРЕНТ} & \textbf{ВТОРОЙ КОНКУРЕНТ} &                   \\ \midrule		\textit{Первоначальные затраты на приобретение}                &                                   &                                         &                                         &                       \\
		Затраты на внедрение                                 & 3.159.633 & 1.500.000 & 3.900.000 & 50                    \\
		Затраты на обучение персонала                        & 631.927 & 250.000 & 800.000 & 15                    \\
		Закупка специального оборудования или каналов связи  & 147.169 & 210.000 & 320.000 & 10                    \\
		\textit{Эксплуатационные затраты}                             &                                   &                                         &                                         &                       \\
		Затраты на обновление и модернизацию                & 315.963 & 300.000 & 240.000 & 15                    \\
		Расходы на управление системой                       & 631.927 & 650.000 & 5.000                                    & 10                    \\
		\textbf{Цена потребления}                            & \textbf{4.886.619}                  & \textbf{2.910.000}                     & \textbf{5.265.000}                      &                       \\ \midrule
		\textbf{Индекс конкурентоспособности по экономическим критериям} & \textbf{0,60} & \textbf{1,00}                           & \textbf{0,55}                            &                       \\
		\bottomrule
	\end{tabular}
	\label{tab:econom_char}
\end{table}