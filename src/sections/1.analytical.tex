\section{Аналитический раздел}
\label{sec:analytics}

\subsection{Актуальность и проблематика цифровой идентификации дипломов}

В современном обществе, дипломы об образовательных достижениях и квалификации служат основным подтверждением учебных успехов и профессиональной компетентности личности. Они являются ключевым фактором при трудоустройстве, продвижении по карьерной лестнице и во многих других аспектах, в том числе социальной жизни. Однако, проблемы, такие как: подделка, утеря или долгий процесс верификации дипломов, становятся всё более актуальными.

Распространение технологий, позволяющих подделывать документы, увеличивает вероятность появления фальшивых дипломов, что затрудняет трудоустройство и создает дополнительные риски для работодателей и общества. Поддельные дипломы наносят ущерб не только репутации учебных заведений, но и снижают качество специалистов в различных сферах, что может привести к длительным негативным последствиям для экономики и общества в целом~\cite{bib:if_fake_diploma, bib:tzh_fake_diploma}.

Процедуры проверки подлинности дипломов часто оказываются времязатратными и сложными, создавая дополнительные трудности для выпускников и работодателей. Такие процедуры включают сверку с базами данных, запросы в учебные заведения и другие методы, направленные на предотвращение использования поддельных документов. Этот процесс может стать препятствием в поиске работы и увеличивает нагрузку на работодателей при отборе кандидатов~\cite{bib:diploma_check}.

В случае утраты или повреждения дипломов возникают серьезные проблемы, связанные с восстановлением утраченной информации, что может негативно повлиять на карьерные перспективы выпускников. Выпускникам приходится сталкиваться с дополнительными сложностями и бюрократическими процедурами для восстановления диплома. Это не только затрудняет процесс трудоустройства, но и может отрицательно сказаться на репутации выпускника в глазах потенциального работодателя~\cite{bib:cp_diploma_reissue, bib:iz_diploma_reissue}.

Внедрение системы токенизации на базе блокчейн-технологий может решить описанные выше проблемы. Токенизация образовательных документов позволяет создавать уникальные, невзаимозаменяемые цифровые слепки для каждого диплома, что значительно усложняет их подделку. Блокчейн, как технология, обеспечивающая децентрализованное хранение и верификацию данных, способна упростить процесс проверки подлинности дипломов, делая его более прозрачным и безопасным. Кроме того, она минимизирует риск утраты важных данных, так как информация будет защищена и всегда доступна в цепочке блоков~\cite{bib:tadviser_digital_diploma}.

Развитие технологий распределенных реестров (Distributed Ledger Technology, DLT)~\cite{bib:distributed_ledger} могут повысить надежность системы высшего образования и адаптировать ее к современным потребностям, способствуя доверию со стороны работодателей и образовательных учреждений.

\subsection{Применение технологий блокчейна в сфере образования}

Распределенные реестры и блокчейн-технологии открывают новые горизонты в сфере образования, обеспечивая децентрализованное, прозрачное и безопасное хранение данных об образовательных достижениях и квалификациях.

Блокчейн, как разновидность технологии распределенных реестров, базируется на принципах перераспределения и коллективного подтверждения валидности данных. В блокчейне данные хранятся в цепочке контейнеров (блоков) для данных, защищенных криптографическими методами. Такой подход обеспечивает устойчивость к фальсификации и несанкционированному доступу, что играет ключевую роль в управлении образовательными документами. Технология обеспечивает надежное хранение данных, облегчает процесс их верификации и уменьшает риски мошенничества~\cite{bib:blockchain_technology}.

Использование блокчейна в образовании находится на начальной стадии, но уже сейчас видны перспективы его развития: распределённый реестр может стать базой для создания унифицированных образовательных платформ, обеспечивающих интероперабельность~\cite{bib:interoperability}, автоматизацию процессов и глобальную доступность информации.

Применение блокчейн-технологий в организации учебного процесса открывает новые возможности. Блокчейн помогает управлять доступом к библиотечным ресурсам, автоматизировать учетные системы и обеспечивать прозрачность использования ресурсов. Регистрация исследовательских данных и публикаций в блокчейн-системе подтверждает их подлинность, обеспечивает прозрачность и защиту интеллектуальной собственности.

Также он упрощает и автоматизирует финансовые операции, такие как оплата обучения, получение грантов и стипендий. С его помощью можно эффективно управлять карьерой, предоставляя студентам и специалистам инструменты для документирования и верификации своих навыков и достижений на протяжении всей профессиональной деятельности, включая верификацию дипломов.

Кроме того, блокчейн служит надежным инструментом для хранения и верификации дипломов, сертификатов и других образовательных документов, снижая риск мошенничества. Академические транскрипты и образовательная история студента могут быть зафиксированы в блокчейне, обеспечивая удобный доступ и передачу данных между учебными заведениями. Блокчейн гарантирует безопасное и прозрачное хранение и обмен данными между студентами и преподавателями, что способствует эффективному управлению данными и соблюдению стандартов конфиденциальности.

\subsection{Блокчейн-технологии и умные контракты для управления образовательными достижениями}

Существует несколько ключевых типов блокчейнов~\cite{bib:types_of_blockchains}, каждый из которых обладает своими уникальными характеристиками и спецификой применения. Основные из них приведены на рисунке~\ref{fig:types_of_blockchains}.

\begin{figure}[H]   
	\centering
	\includegraphics[width=\textwidth]{images/1.types_of_blockchains.png}
	\parskip=6pt
	\caption{Ключевые типы блокчейнов}
	\label{fig:types_of_blockchains}
\end{figure}
Публичные блокчейны представляют собой открытые децентрализованные системы, в которых каждый участник имеет доступ к процессу верификации и валидации транзакций. Эти системы характеризуются высоким уровнем прозрачности и демократичности, позволяя каждому участнику сети взаимодействовать, консенсусно~\cite{bib:consensus} подтверждать и проверять записи в блокчейне. Примером публичного блокчейна является сеть Ethereum~\cite{bib:ethereum}.

Частные блокчейны, напротив, являются закрытыми системами с тщательно регулируемым уровнем доступа и правами участников. В таких системах централизованный администратор или определенная группа лиц контролирует процесс верификации и валидации транзакций, что обеспечивает конфиденциальность и защиту данных. Частные блокчейны часто используются в корпоративных и конфиденциальных системах.

Консорциумные блокчейны представляют собой гибридный вариант, сочетающий элементы публичных и частных блокчейнов. Управление такой сетью осуществляется несколькими определенными организациями, что обеспечивает сбалансированный уровень доверия, децентрализации и контроля в системе. Этот тип блокчейна применяется в межорганизационных проектах.

Применительно к NFT-дипломам, стоит выбирать консорциумные или публичные блокчейны, поскольку в частном блокчейне отсутствует требуемый уровень децентрализации и верификации со стороны независимых участников, что дает возможность организаторам учебного процесса фальсифицировать данные без возможности проведения независимой проверки. Подробное сравнение приведено в таблице~\ref{tab:blockchain_comparison}.

\begin{table}[H]
    \caption{Сравнительная таблица типов блокчейнов для NFT-дипломов}
    \centering

    \tolerance=0
    \emergencystretch=10pt
    \hyphenpenalty=0
    \exhyphenpenalty=0
    \begin{tabular}{|x{5.6cm}|x{3cm}|x{3cm}|x{3cm}|}
        \hline
        \textbf{Характеристика} & \textbf{Публичный блокчейн} & \textbf{Частный блокчейн} & \textbf{Консорциумный блокчейн} \\ \hline
        Уровень децентрализации & Высокий & Низкий & Средний \\ \hline
        Прозрачность & Высокая & Низкая & Средняя \\ \hline
        Доступность & Открытый доступ для всех & Доступ ограничен администраторами & Доступ ограничен несколькими организациями \\ \hline
        Безопасность & Высокая, за счет децентрализации & Высокая, за счет контроля доступа & Высокая, за счет совместного управления \\ \hline
        Конфиденциальность & Низкая & Высокая & Средняя \\ \hline
        Скорость транзакций & Ниже, зависит от нагрузки сети & Высокая, благодаря контролируемому доступу & Средняя, баланс между скоростью и децентрализацией \\ \hline
        Масштабируемость & Ограниченная & Высокая & Средняя \\ \hline
        Примеры использования & Ethereum, Siberium & Hyperledger Fabric & R3 Corda, Мастерчейн \\ \hline
        Подходит для NFT-дипломов & Да, благодаря прозрачности и независимой верификации & Нет, из-за отсутствия полноценной децентрализации & Да, благодаря сбалансированному уровню доверия и контроля \\ \hline
    \end{tabular}
    \label{tab:blockchain_comparison}
\end{table}

Одним из ключевых преимуществ блокчейн-технологий является надежность и защищенность систем. Использование криптографического шифрования обеспечивает защиту данных, сохраняя подлинность и целостность образовательных документов.

Благодаря распределенной сети блоков минимизируются риски потери данных из-за сбоев в работе или целенаправленных атак на централизованные серверы. В таблице~\ref{tab:blockchain_in_education} приведен подробный анализ, который сопоставляет преимущества и недостатки применения блокчейна в образовании.

\begin{table}[H]
    \caption{Преимущества и недостатки применения блокчейн-технологий в образовании}
    \centering

    \tolerance=0
    \emergencystretch=10pt
    \hyphenpenalty=0
    \exhyphenpenalty=0
    \begin{tabular}{|x{3cm}|x{6cm}|x{6cm}|}
        \hline
        \textbf{Параметр} & \textbf{Преимущество} & \textbf{Недостаток} \\ \hline
        Прозрачность и Верификация & Блокчейн обеспечивает прозрачность и возможность верификации академических данных, что уменьшает вероятность мошенничества с документами и подтверждает подлинность дипломов. & Избыток прозрачности может вызвать проблемы с конфиденциальностью данных, если не реализованы соответствующие механизмы защиты. \\ \hline
        Устойчивость к Изменениям & Благодаря децентрализации блокчейн, зарегистрированные данные не могут быть изменены или удалены, что обеспечивает высокую степень защиты от фальсификаций. & В то же время, это может создать проблемы, если потребуется корректировка данных по каким-то причинам. \\ \hline
        Глобальное Признание и Переносимость & Блокчейн-сертификаты могут быть легко доступны и признаны на международном уровне, облегчая мобильность студентов и узнаваемость вуза. & Отсутствие глобальных стандартов и регулирующих рамок может вызвать проблемы с признанием и использованием блокчейн-дипломов в разных странах и учебных заведениях. \\ \hline
        Непрерывность доступа к данным & Блокчейн обеспечивает постоянный доступ к данным, что минимизирует риски их утраты из-за сбоев в работе серверов или атак. & Возможны проблемы с масштабируемостью сети при значительном увеличении объема данных и количества пользователей. \\ \hline
        Интеграция с существующими системами & Блокчейн может быть интегрирован с существующими системами управления данными, улучшая их безопасность и эффективность. & Трудности интеграции могут возникнуть из-за несовместимости технологий и необходимости модернизации существующей инфраструктуры. \\ \hline
    \end{tabular}
    \label{tab:blockchain_in_education}
\end{table}

Основываясь на собранных данных, можно заключить, что внедрение технологии блокчейн для NFT-дипломов в образовательной сфере представляет собой значительный шаг в сторону модернизации и улучшения управления академическими данными. Преимущества, такие как прозрачность, верификация, устойчивость к фальсификации, децентрализация, экономия ресурсов и глобальное признание, могут значительно повысить эффективность и надежность процессов выдачи и верификации дипломов. Это, в свою очередь, может улучшить репутацию учебных заведений на международном уровне.

Хотя некоторые недостатки, такие как отсутствие глобальных стандартов, требуют внимания, преимущества, которые блокчейн может принести в образовательную сферу, делают его привлекательным вариантом для внедрения NFT-дипломов в высших учебных заведениях.

Умные контракты --- это самоисполняющиеся программы, предназначенные для автоматического выполнения и верификации сделок или иных действий при наступлении заранее определенных условий. В контексте образовательного процесса умные контракты могут существенно способствовать автоматизации, прозрачности и надежности процедур верификации и учета академических достижений~\cite{bib:smart_contract}.

Применение умных контрактов позволяет минимизировать влияние человеческого фактора и снижает риск возникновения ошибок или манипуляций с данными со стороны недобросовестных участников процесса. Эти контракты могут автоматизировать процессы выдачи и проверки подлинности дипломов, сертификатов и других образовательных документов.

К сожалению, использование умных контрактов сопряжено с рядом сложностей и ограничений. Во-первых, для разработки, внедрения и поддержки умных контрактов требуется глубокая техническая экспертиза. Во-вторых, умные контракты функционируют строго в соответствии с заложенной в них логикой, что ограничивает их гибкость и адаптивность к изменяющимся обстоятельствам или нестандартным ситуациям без дополнительных модификаций.

\subsection{Инструменты токенизации и виды токенов}

Токенизация~\cite{bib:tokenizer} представляет собой процесс преобразования прав на физический или цифровой актив в цифровой токен на блокчейне. В контексте образования токенизация может использоваться для управления академическими достижениями, дипломами и сертификатами. Существует несколько видов токенов, каждый из которых обладает уникальными характеристиками и применениями.

NFT (Non-Fungible Tokens)~\cite{bib:what_is_nft} --- это уникальные цифровые токены, которые представляют собой собственность на определенный объект или актив. В образовательной сфере NFT могут быть использованы для подтверждения подлинности и уникальности дипломов, сертификатов и других академических достижений.

Преимущества и недостатки:
\begin{itemize}
    \item каждый NFT представляет собой уникальный актив, что исключает возможность подделки;
    \item все транзакции с NFT записываются в блокчейн, обеспечивая прозрачность и легкость верификации;
    \item NFT могут быть легко переданы от одного владельца к другому, упрощая процесс обмена и проверки документов;
    \item создание и управление NFT требует значительных вычислительных ресурсов и может быть дорогостоящим;
    \item не все платформы и блокчейны поддерживают NFT, что может создавать барьеры для их широкого использования.
\end{itemize}

SBT (Soulbound Tokens)~\cite{bib:what_is_sbt} --- это непередаваемые токены, привязанные к личности владельца. Они предназначены для хранения и подтверждения личных данных и достижений, которые не могут быть переданы или проданы.

Преимущества и недостатки:
\begin{itemize}
    \item SBT гарантирует, что токены не могут быть переданы или проданы, что делает их идеальными для подтверждения личных академических достижений;
    \item поскольку SBT не могут быть переданы, они снижают риск мошенничества и подделки документов;
    \item невозможность передачи токенов ограничивает их применение в ситуациях, требующих обмена данными между различными субъектами;
    \item поскольку SBT привязаны к личности, потеря доступа к ним может привести к утрате важных данных.
\end{itemize}

Утилитарные токены~\cite{bib:what_is_utt} --- это цифровые токены, которые предоставляют доступ к определенным услугам или функциональности внутри определенной экосистемы. В образовательной сфере utility токены могут использоваться для доступа к образовательным ресурсам, оплате курсов или участию в обучающих программах.

Преимущества и недостатки:
\begin{itemize}
    \item утилитарные токены могут использоваться для доступа к различным образовательным ресурсам и услугам;
    \item эти токены могут быть использованы в различных контекстах и для различных целей, что делает их универсальными;
    \item утилити токены имеют ограниченное применение за пределами своей экосистемы;
    \item стоимость токенов может быть подвержена волатильности, что создает финансовые риски для пользователей.
\end{itemize}

\subsection{Система децентрализованного хранения данных (IPFS)}

Межпланетная файловая система (IPFS)~\cite{bib:ipfs, bib:ipfs_is} это децентрализованная технология хранения данных. IPFS функционирует как распределенная файловая система, позволяющая хранить и обмениваться файлами в одноранговой сети. Эта заменяет централизованные серверова, распределяя данные по множеству узлов сети, что повышает устойчивость к сбоям и атакам и снижает риск потери данных. Принцип работы HTTP и IPFS приведен на рисунке~\ref{fig:http_vs_ipfs}.

\begin{figure}[H]   
	\centering
	\includegraphics[width=\textwidth]{images/1.http_vs_ipfs.png}
	\parskip=6pt
	\caption{Сравнение HTTP с IPFS обменом данных}
	\label{fig:http_vs_ipfs}
\end{figure}

Одной из ключевых функций IPFS является контент-адресация~\cite{bib:ipfs_2}, где данные идентифицируются не по их расположению, а по содержимому, используя уникальные хэш-коды. Это обеспечивает неизменность данных и облегчает их верификацию, что особенно важно для хранения академических документов. Кроме того, IPFS использует дедупликацию~\cite{bib:dedup}, храня одинаковые данные только один раз, что экономит пространство и ресурсы, снижая дублирование данных. Масштабируемость системы позволяет легко расширяться по мере добавления новых узлов и данных, что обеспечивает возможность обработки больших объемов информации без потери производительности.

В образовательной сфере IPFS может быть использована для хранения студенческих записей, дипломов, сертификатов и других важных академических документов. Такое решение обеспечивает надежность и долговечность хранения данных, минимизируя риск их утраты или повреждения. Преподаватели и студенты могут легко обмениваться учебными материалами и результатами исследований, что упрощает доступ к ресурсам и улучшает сотрудничество между участниками образовательного процесса.

\subsection{Анализ существующих систем}

Одной из первых организаций, внедривших блокчейн в образование, стал Массачусетский технологический институт (MIT). Выпускники MIT могут выбрать получение цифровой версии диплома бесплатно. Этот диплом отправляется в виде вложения к электронному письму, закодированного и поддающегося проверке с помощью технологии Blockcerts. Разработанная в сотрудничестве с компанией Learning Machine, эта технология использует блокчейн для защиты и верификации дипломов~\cite{bib:mit_diplomas}.

Ещё одним примером внедрения блокчейн-технологий является проект социальной сети ВКонтакте (VK), направленный на использование открытого блокчейна Polygon~\cite{bib:polygon} для выдачи NFT-дипломов. ВКонтакте разработала систему, позволяющую университетам выдавать цифровые дипломы и сертификаты в форме NFT не покидая социальной сети~\cite{bib:vk_nft_diploma}.

Московский физико-технический институт (МФТИ) также активно применяет блокчейн-технологии в образовательном процессе. МФТИ разработал и внедрил систему на основе Ethereum и один из первых в России выдал студентам NFT. Дипломы были выданы через платформу OpenSea, которая заблокирована на территории РФ по решению суда~\cite{bib:opensea_block}, поэтому увидеть токены можно только через другие системы~\cite{bib:mipt_nft_diploma}.

Студенты, прошедшие серию инженерных курсов Duke через платформу онлайн-обучения Coursera, в 2022 году получили копию своего сертификата в виде невзаимозаменяемого токена. Сертификаты NFT ценны для студентов как свидетельство их академической успеваемости. Ещё тогда преподаватель курса говорил, что это, вероятно, будущее сертификации образования~\cite{bib:duke_nft_diploma}.

В таблице~\ref{tab:nft_diplomas_comparison} приведено общее сравнение предлагаемых решений. Существующие системы сравниваются по ряду параметров для представления рынка.

\begin{table}[H]
    \caption{Cравнительная таблица систем NFT-дипломов}
    \centering

    \tolerance=0
    \emergencystretch=10pt
    \hyphenpenalty=0
    \exhyphenpenalty=0
    \begin{tabular}{|x{4cm}|x{3.2cm}|x{3.2cm}|x{3.2cm}|}
        \hline
        \textbf{Параметр} & \textbf{MIT} & \textbf{ВКонтакте (VK)} & \textbf{МФТИ} \\ \hline
        Стоимость выпуска и владения & Высокие начальные затраты, низкие эксплуатационные & Средние затраты на выпуск, низкие эксплуатационные & Высокие начальные и эксплуатационные затраты \\ \hline
        Применяемый блокчейн & Публичный блокчейн, Blockcerts & Публичный блокчейн, Polygon & Публичный блокчейн, Bitcoin \\ \hline
        Вид токена & SBT & SBT & NFT \\ \hline
        Прозрачность и контроль & Возможность публичного аудита & Публичный аудит & Публичный аудит, управление доступом \\ \hline
        Удобство использования & Интеграция в личный кабинет & Интеграция в профиль & Публичная сеть блокчейн \\ \hline
        Конфиденциальность & Высокий & Нет защиты персональных данных & Высокий \\ \hline
        Интероперабельность & Низкая & Средняя & Низкая \\ \hline
        Пользовательский опыт & Удобный интерфейс для администраторов и студентов & Интуитивный интерфейс, интеграция с соцсетями & Интеграция с существующими LMS \\ \hline
        Модульность & Возможность добавления новых модулей и функций & Ограниченные возможности для расширения & Высокая возможность для добавления новых функций \\ \hline
    \end{tabular}
    \label{tab:nft_diplomas_comparison}
\end{table}

