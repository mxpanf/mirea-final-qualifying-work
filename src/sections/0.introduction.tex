\supersection{Введение}
\label{sec:introduction}

В условиях стремительного развития цифровых технологий система образования переживает значительные изменения, требующие повышение эффективности и прозрачности образовательных процессов. Одним из ключевых аспектов, доступных для модернизации, является система верификации и хранения образовательных достижений, включая дипломы и сертификаты. Традиционные методы проверки подлинности документов сталкиваются с множеством проблем, таких как подделка, утрата и сложные бюрократические процедуры, что затрудняет трудоустройство выпускников и снижает доверие к образовательным учреждениям.

Распространение технологий, позволяющих подделывать документы, увеличивает вероятность появления фальшивых дипломов, что создает дополнительные риски для работодателей и общества в целом. Поддельные дипломы наносят ущерб не только репутации учебных заведений, но и снижают качество специалистов в различных сферах, что может привести к длительным негативным последствиям для экономики и социальной жизни. Процедуры проверки подлинности дипломов часто оказываются времязатратными, создавая дополнительные трудности для выпускников и работодателей. В случае утраты или повреждения дипломов возникают серьезные проблемы, что может негативно повлиять на карьерные перспективы.

На этом фоне блокчейн-технологии и смарт-контракты становятся оптимальными инструментами для решения проблем. Блокчейн предоставляет возможность децентрализованного хранения данных, обеспечивая неизменность и доступность, упрощая процесс проверки дипломов. Умные контракты, позволяют автоматизировать процессы выдачи, снижая риск человеческих ошибок и увеличивая скорость и прозрачность процедур.

Целью данной работы является разработка системы токенизации дипломов на базе блокчейн-технологий для цифровой верификации и децентрализованного хранения образовательных достижений. В рамках исследования рассматриваются функциональные и нефункциональные требования к системе, архитектура, клиенты в виде веб-приложения и Telegram-бота, процесс выпуска и верификации цифровых дипломов, а также анализ существующих решений и выбор подходящего стека технологий.

Актуальность данного исследования обусловлена растущей потребностью в надежных и эффективных системах управления образовательными данными, способных обеспечивать высокую степень защиты и доверия. Разработка и внедрение системы токенизации дипломов на базе блокчейн-технологий представляют собой значительный шаг в сторону модернизации и улучшения управления образовательными достижениями, что существенно повысит доверие к образовательным учреждениям и упростит процесс трудоустройства выпускников.